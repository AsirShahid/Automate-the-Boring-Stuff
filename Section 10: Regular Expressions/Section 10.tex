% Created 2021-08-03 Tue 20:35
% Intended LaTeX compiler: pdflatex
\documentclass[11pt]{article}
\usepackage[utf8]{inputenc}
\usepackage[T1]{fontenc}
\usepackage{graphicx}
\usepackage{grffile}
\usepackage{longtable}
\usepackage{wrapfig}
\usepackage{rotating}
\usepackage[normalem]{ulem}
\usepackage{amsmath}
\usepackage{textcomp}
\usepackage{amssymb}
\usepackage{capt-of}
\usepackage{hyperref}
\author{Mohammed Asir Shahid}
\date{2021-08-03}
\title{Section 10\\\medskip
\large Regular Expressions}
\hypersetup{
 pdfauthor={Mohammed Asir Shahid},
 pdftitle={Section 10},
 pdfkeywords={},
 pdfsubject={},
 pdfcreator={Emacs 27.2 (Org mode 9.5)}, 
 pdflang={English}}
\begin{document}

\maketitle
\tableofcontents


\section{Regular Expression Basics}
\label{sec:org2d2c63c}

In this lesson, we will be working with pattern matching and regular expressions. Regular expressions allow you to specify a pattern of text to search for.

An example of a text pattern would be a phone number. 415-555-0000. In the US, this is the standard way of writing up phone numbers. If we had that same number but without the hyphens, we would not recognize it as a phone number.

\begin{verbatim}
def isPhoneNumber(text):
    if len(text) != 12:
        return False
    for i in range(0, 3):
        if not text[i].isdecimal():
            return False
    if text[3] != '-':
        return False
    for i in range(4, 7):
        if not text[i].isdecimal():
            return False
    if text[7] != '-':
        return False
    for i in range(8, 12):
        if not text[i].isdecimal():
            return False
    return True

print(isPhoneNumber("415-555-1234"))
print(isPhoneNumber("My Phone Number"))

\end{verbatim}

\begin{verbatim}
True
False
\end{verbatim}


That's a lot of code for a relatively simple task. If we want to find phone numbers in large strings, we'd need to write some more code.

\begin{verbatim}
def isPhoneNumber(text):
    if len(text) != 12:
        return False
    for i in range(0, 3):
        if not text[i].isdecimal():
            return False
    if text[3] != '-':
        return False
    for i in range(4, 7):
        if not text[i].isdecimal():
            return False
    if text[7] != '-':
        return False
    for i in range(8, 12):
        if not text[i].isdecimal():
            return False
    return True

print(isPhoneNumber("415-555-1234"))
print(isPhoneNumber("My Phone Number"))

message="Call me at 415-555-1011 tomorrow, or 415-555-9999 for my office line"

foundNuumber=False

for i in range(len(message)):
    chunk=message[i:i+12]
    if isPhoneNumber(chunk) == True:
        print("Phone number found")
        foundNumber=True
if not foundNumber:
    print("Could not find any phone numbers")
    
\end{verbatim}

\begin{verbatim}
True
False
Phone number found
Phone number found
\end{verbatim}

\subsection{The re Module}
\label{sec:org33c94d6}

We can write the previous code much faster using regular expressions.

\begin{verbatim}

import re

message="Call me at 415-555-1011 tomorrow, or 415-555-9999 for my office line"

phoneNumRegex=re.compile(r"\d\d\d-\d\d\d-\d\d\d\d")

mo=phoneNumRegex.search(message)

print(type(mo))
print(mo.group())

print(phoneNumRegex.findall(message))


\end{verbatim}

\begin{verbatim}
<class 're.Match'>
415-555-1011
['415-555-1011', '415-555-9999']
\end{verbatim}
\end{document}
