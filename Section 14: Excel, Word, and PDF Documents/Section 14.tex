% Created 2021-08-05 Thu 19:19
% Intended LaTeX compiler: pdflatex
\documentclass[11pt]{article}
\usepackage[utf8]{inputenc}
\usepackage[T1]{fontenc}
\usepackage{graphicx}
\usepackage{grffile}
\usepackage{longtable}
\usepackage{wrapfig}
\usepackage{rotating}
\usepackage[normalem]{ulem}
\usepackage{amsmath}
\usepackage{textcomp}
\usepackage{amssymb}
\usepackage{capt-of}
\usepackage{hyperref}
\author{Mohammed Asir Shahid}
\date{2021-08-05}
\title{Section 14\\\medskip
\large Excel, Word, and PDF Documents}
\hypersetup{
 pdfauthor={Mohammed Asir Shahid},
 pdftitle={Section 14},
 pdfkeywords={},
 pdfsubject={},
 pdfcreator={Emacs 27.2 (Org mode 9.5)}, 
 pdflang={English}}
\begin{document}

\maketitle
\tableofcontents


\section{Reading Excel Spreadsheets}
\label{sec:org2da0ac6}

The openpyxl module lets us modify Excel files using Python. It is a third party module that we'll need to install ourselves.

\begin{verbatim}
pip install openpyxl
\end{verbatim}

The Excel document is called a workbook that is saved by .xlsx file extension. Each workbook contains sheets/worksheets. Inside each sheet there are columns (letters) and rows (numbers). The intersection of a column and row is called a cell.


\begin{verbatim}

import openpyxl,os

workbook=openpyxl.load_workbook("example.xlsx")

print(type(workbook))

print(workbook.get_sheet_names())

sheet=workbook.get_sheet_by_name("Sheet1")
print(type(sheet))

cell=sheet["A1"]

print(type(cell))
print(cell.value)

cell=sheet["B1"]

print(type(cell))
print(cell.value)

cell=sheet["C1"]

print(type(cell))
print(cell.value)


print(type(cell))
print(cell.value)

for i in range(1,8):
    print(i, sheet.cell(row=i, column=2).value)

\end{verbatim}

\begin{verbatim}
<class 'openpyxl.workbook.workbook.Workbook'>
['Sheet1', 'Sheet2', 'Sheet3']
<class 'openpyxl.worksheet.worksheet.Worksheet'>
<class 'openpyxl.cell.cell.Cell'>
2015-04-05 13:34:02
<class 'openpyxl.cell.cell.Cell'>
Apples
<class 'openpyxl.cell.cell.Cell'>
73
<class 'openpyxl.cell.cell.Cell'>
73
1 Apples
2 Cherries
3 Pears
4 Oranges
5 Apples
6 Bananas
7 Strawberries
\end{verbatim}
\end{document}
