% Created 2021-07-25 Sun 18:50
% Intended LaTeX compiler: pdflatex
\documentclass[11pt]{article}
\usepackage[utf8]{inputenc}
\usepackage[T1]{fontenc}
\usepackage{graphicx}
\usepackage{grffile}
\usepackage{longtable}
\usepackage{wrapfig}
\usepackage{rotating}
\usepackage[normalem]{ulem}
\usepackage{amsmath}
\usepackage{textcomp}
\usepackage{amssymb}
\usepackage{capt-of}
\usepackage{hyperref}
\author{Mohammed Asir Shahid}
\date{2021-07-23}
\title{Section 1\\\medskip
\large Python Basics}
\hypersetup{
 pdfauthor={Mohammed Asir Shahid},
 pdftitle={Section 1},
 pdfkeywords={},
 pdfsubject={},
 pdfcreator={Emacs 27.2 (Org mode 9.5)}, 
 pdflang={English}}
\begin{document}

\maketitle
\tableofcontents


\section{Basic Terminology}
\label{sec:orgd833fe7}


\begin{verbatim}
print(2+2)
\end{verbatim}

\begin{verbatim}
4
\end{verbatim}


The above is an expression consisting of operators (such as the +) and values (such as the 2). Expressions always evaluate down to a single value.

\begin{verbatim}
print(2)
print(5-3)
print(3*7)
print(22/7)
\end{verbatim}

\begin{verbatim}
2
2
21
3.142857142857143
\end{verbatim}


The order of operations follow PEMDAS, so we can use parentheses in order to show what we really want.


\begin{verbatim}
print(2+3*6)
print((2+3)*6)
\end{verbatim}

\begin{verbatim}
20
30
\end{verbatim}


We can get errors when we mess things up

\begin{verbatim}
print(5+)
\end{verbatim}

We can have strings and they can also be concatenated or replicated.

\begin{verbatim}
print("Hello World")
print("Alice" + "Bob")
print("Alice"*3)
print("Hello" + "!"*10)
\end{verbatim}

\begin{verbatim}
Hello World
AliceBob
AliceAliceAlice
Hello!!!!!!!!!!
\end{verbatim}


Python can store values inside of variables. Usually it is smart to name variables that describe the values that they contain. Variables can also be overwritten easily.

\begin{verbatim}
spam=42
print(spam)
spam="Hello"
print(spam)
print(spam + " World")
\end{verbatim}

\begin{verbatim}
42
Hello
Hello World
\end{verbatim}


Variables can also be set to different expressions. The expression will be evaluated and then that value will be set to the variable.

\begin{verbatim}
spam=2+2
print(spam)
spam=10
print(spam)
spam=spam+1
print(spam)
\end{verbatim}

\begin{verbatim}
4
10
11
\end{verbatim}

\section{Writing our First program}
\label{sec:org892bd00}

\begin{verbatim}
# When using Python outside of org mode, we use a different way to get inputs
# name=input()

print("Hello World")

print("What is your name?")

print("It is good to meet you, {}".format(name))
# Alternatively, we could concatenate
# print("It is good to meet you, " + name)
print("The length of your name is: {}".format(str(len(name))))

print("You will be {} in a year".format(str(int(age)+1)))
\end{verbatim}

\begin{verbatim}
Hello World
What is your name?
It is good to meet you, Asir
The length of your name is: 4
You will be 23 in a year
\end{verbatim}
\end{document}
